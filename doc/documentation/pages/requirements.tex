\newpage
\section{Anforderungen}\label{sec:requirements}

\subsection{Must-Have}
Folgende Aspekte müssen schlussendlich in der App implementiert sein:
\begin{itemize}
    \item Die Eingabe des zu übersetzenden Texte durch Spracheingabe, welche auch ohne funktionierende Internetverbindung, die Sprache zu Text umwandelt.
    \item Die Übersetzung des eingegebenen Textes mit einer Bibliothek, welche auch ohne bestehende Internetverbindung den Text übersetzt.
    \item     Die Speicherung des Sprachverlaufs: Zuvor eingesprochene Texte sollen mit
ihrer Übersetzung gespeichert werden, damit diese zu einem späteren
Zeitpunkt weiterhin abrufbar sind.
    \item Die Ausgabe des übersetzten Textes auf dem Display.
\end{itemize}

\subsection{Should-Have}
Folgende Aspekte sollten schlussendlich in der App implementiert sein:
\begin{itemize}
    \item Dies Ausgabe des übersetzten Textes in Form einer Sprachausgabe, welche auch ohne funktionierende Internetverbindung, den Text zu Sprache umwandelt.
    \item Das Übersetzen des Textes mithilfe eines neuronalen Netzes, wie zum Beispiel
GPT-JT oder GPT J-6B, welches auf einem lokalen Server ausgeführt wird.
\end{itemize}

\subsection{Could-Have}
Folgende Aspekte können schlussendlich in der App implementiert sein:
\begin{itemize}
    \item Bei einer bestehenden Internetverbindung kann zusätzlich ausgewählt werden, dass ChatGPT oder DeepL zur Übersetzung des Textes verwendet wird.
\item Die nachträgliche Aufbereitung des Textes durch DeepL Write oder ChatGPT, sobald eine Internetverbindung besteht.
\item App-Einstellungen zum Auswählen der Sprachen, eines Darkmodes oder ähnlichen Optionen.
\item Die Ausgabe der Übersetzungen in Form einer Datei (z.B. PDF)
\end{itemize}