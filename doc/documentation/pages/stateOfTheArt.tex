\newpage
\section{Technologien (WIP)}\label{sec:stateOfTheArt}
In folgendem Abschnitt werden die genutzten Technologien aufgelistet, sowie ihre Vor- und Nachteile in Bezug zu unserem Projekt diskutiert.
\subsection{Sprachein- und Ausgabe (WIP)}

\subsubsection{Android - SpeechRecognizer (WIP)}
Der Android SpeechRecognizer ermöglicht es uns Texte per Spracheingabe aufzunehmen.

\subsubsection{Android - TextToSpeech (WIP)}
Android TextToSpeech ermöglicht es uns Texte per Sprachausgabe wiederzugeben.

\subsection{Übersetzung (WIP)}

\subsubsection{Google ML Kit - Übersetzer}
Das ML Kit von dem wir den Übersetzer benutzen ist ein von Google entwickeltes Software Development Kit. Es ermöglicht uns Übersetzungsfunktionen mit maschinellen Lernen in unsere App zu integrieren. Es bietet eine benutzerfreundliche Schnittstelle und ist für mobile Anwendungen optimiert.

\subsubsection{DeepL - Übersetzer}
Der DeepL Übersetzer ist ein Übersetzungstool auf das hauptsächlich über die zugehörige Website bzw. Anwendung benutzt wird. Wir nutzen DeepL in unserem Projekt als sekundäre Übersetzung, sobald eine Internetverbindung besteht, da dieser Übersetzer nicht offline genutzt werden kann. Aber somit können wir bei bestehender Verbindung eine sehr gute Übersetzung sicherstellen und schon übersetzte Texte rückwirkend auf Fehler überprüfen.