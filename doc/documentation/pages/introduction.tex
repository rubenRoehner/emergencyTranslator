\newpage
\section{Einleitung}\label{sec:introduction}

\subsection{Motivation}
Mit steigender Diversität der Gesellschaft kommt es immer öfters vor, dass Personen, mit Migrationshintergrund in eine Notlage geraten und dann in dieser Stresssituation Schwierigkeiten beim Kommunizieren in einer Fremdsprache haben.

Herkömmliche Übersetzer-Apps verfügen teilweise schon über Unterhaltungsfunktionen, sind dabei aber auf eine bestehende stabile Internetverbindung angewiesen. Darüber hinaus wird in keiner dieser Apps der Sprachverlauf angezeigt, sondern nur der letzte gesprochene Satz. Gerade in Notsituationen, wo man das Übersetzte nochmal zur Sicherheit prüfen möchte, kann dies zu Unsicherheit und Stress führen.

Unsere Notfall-Übersetzer-App löst diese bestehenden Probleme.

\subsection{Ziel}
In diesem Projekt soll eine App entwickelt werden, die auch ohne bestehende Internetverbindung funktioniert. Dies hilft zum einen in Regionen mit schlechtem Internetempfang. So kann die App beispielsweise auch nach einem Autounfall auf dem Land genutzt werden. Außerdem funktioniert die App so auch bei einem zentralen Stromausfall.

Darüber hinaus verfügt unsere App auch über eine leicht verständliche Sprachaufzeichnung. Das hilft, wichtige Details nicht zu vergessen. Das ist zum Beispiel in der Notaufnahme wichtig, wo man noch einmal in Ruhe nachlesen kann, was der Patient gesagt hat und im Zweifelsfall die Übersetzungen noch einmal überprüfen kann.