\newpage
\section{Fazit}\label{sec:conclusion}
Es ist uns gelungen, alle Must-Have Anforderungen umzusetzen und sogar Funktionen aus den Kategorien Should-Have und Could-Have, wie die Sprachausgabe oder ein Einstellungsmenü, in die App zu integrieren. Die Übersetzung durch ein gehostetes neuronales Netz konnten wir wegen mangelnder Zeit nicht mehr umsetzen, weshalb wir den Einstellungsbereich aus den Could-Have Anforderungen vorgezogen haben.

Obwohl die Anwendung nur außerhalb der vorgesehenen Notfallsituationen getestet wurde, zeigen die durchgeführten Tests, dass die Anwendung leicht verständlich ist und grundsätzlich funktioniert. 
Das größte Problem ist derzeit die Spracherkennung durch Vosk, da man sehr deutlich sprechen muss, um keine Fehler zu generieren.

Darüber hinaus bietet dieses Projekt sehr viele Erweiterungsmöglichkeiten. So wäre es sinnvoll, einen Modus einzubinden, der es ermöglicht, Online-Dienste wie DeepL zu verwenden, wenn eine Internetverbindung besteht. Es wäre auch interessant zu untersuchen, ob und wie ein komplexeres Spracherkennungsmodell auf der geringen Leistung eines Smartphones implementiert werden kann.